\documentclass{aastex63}
\usepackage[utf8]{inputenc}

\usepackage{amsmath}

\newcommand*{\wasyfamily}{\fontencoding{U}\fontfamily{wasy}\selectfont}
\newcommand*{\astrosun}{{\odot}}
\newcommand*{\mercury}{{\text{\wasyfamily\char39}}}
\newcommand*{\venus}{{\text{\wasyfamily\char25}}}
\newcommand*{\mars}{{\text{\wasyfamily\char26}}}
\newcommand*{\jupiter}{{\text{\wasyfamily\char88}}}
\newcommand*{\saturn}{{\text{\wasyfamily\char89}}}
\newcommand*{\uranus}{{\text{\wasyfamily\char90}}}
\newcommand*{\neptune}{{\text{\wasyfamily\char91}}}
\newcommand*{\pluto}{{\text{\wasyfamily\char92}}}

\begin{document}

\title{The Direct Mid-Infrared Detectability of Habitable-zone Exoplanets Around Nearby Stars}


\author{Zach Werber}
\affiliation{Steward Observatory, University of Arizona, Tucson, AZ}

\section{\label{sec:intro}Question Answers}
\subsection{Q1}
The total mass of M31 and the Milky Way are equal in this simulation and the dark matter halo dominates the mass of each.

\subsection{Q2}
M31 has more stellar mass (1.635 times) in this simulation and thus we expect it to be more luminous.

\subsection{Q3}
The Milky Way has 1.028 times the Dark Matter mass as M31. You might think that M31 should have more dark matter mass since it has more stellar mass, but the mass distribution and structure allows this to work as a gravitationally bound system.

\subsection{Q4}
The Baryon fraction is shown in Table 1 as 0.041, 0.067, and 0.046 for the Milky Way, M31, and M33, respectively. These all hover around $\sim$ 4 times lower than the ratio found in the Universe. To account for this difference, we must look outside of galaxies. The difference likely comes from the gas or particles making up the intergalactic medium. While there may not be much in any given region, in an infinitely expanding universe, there is so much space that the Baryon fraction of the universe becomes much higher when accounting for this gas.

\begin{deluxetable*}{cccccc}
\tablenum{1}
\tablecaption{HW3 masses of elements of the 3 galaxies \label{tab:galaxydata}}
\tablewidth{0pt}
\tablehead{
\colhead{Galaxy} & \colhead{Halo Mass ($10^{12} M_\odot$)} & \colhead{Disk Mass ($10^{12} M_\odot$)} & \colhead{Bulge Mass ($10^{12} M_\odot$)} &
\colhead{Total Mass ($10^{12} M_\odot$)} & \colhead{$f_{bar}$} 
}
\decimalcolnumbers
\startdata
Milky Way   & 1.975    & 0.075  & 0.01 & 2.06  & 0.041       \\
Andromeda (M31)  & 1.921  & 0.12 & 0.019     & 2.06   & 0.067      \\
Triangulum (M33) & 0.187 & 0.009     & 0      & 0.196     & 0.046      \\
Local Group & 4.083    & 0.204   & 0.029   & 4.316 & 0.054      \\
\enddata

\end{deluxetable*}


\end{document}